% !TEX TS-program = pdflatex
% !TEX encoding = UTF-8 Unicode

% This is a simple template for a LaTeX document using the "article" class.
% See "book", "report", "letter" for other types of document.

\documentclass[11pt]{report} % use larger type; default would be 10pt

\usepackage[utf8]{inputenc} % set input encoding (not needed with XeLaTeX)

%%% Examples of Article customizations
% These packages are optional, depending whether you want the features they provide.
% See the LaTeX Companion or other references for full information.

%%% PAGE DIMENSIONS
\usepackage{geometry} % to change the page dimensions
\geometry{a4paper} % or letterpaper (US) or a5paper or....
% \geometry{margin=2in} % for example, change the margins to 2 inches all round
% \geometry{landscape} % set up the page for landscape
%   read geometry.pdf for detailed page layout information

\usepackage{graphicx} % support the \includegraphics command and options

% \usepackage[parfill]{parskip} % Activate to begin paragraphs with an empty line rather than an indent

%%% PACKAGES
\usepackage{booktabs} % for much better looking tables
\usepackage{array} % for better arrays (eg matrices) in maths
\usepackage{paralist} % very flexible & customisable lists (eg. enumerate/itemize, etc.)
\usepackage{verbatim} % adds environment for commenting out blocks of text & for better verbatim
\usepackage{subfig} % make it possible to include more than one captioned figure/table in a single float
\usepackage{enumitem}

% These packages are all incorporated in the memoir class to one degree or another...

%%% HEADERS & FOOTERS
\usepackage{fancyhdr} % This should be set AFTER setting up the page geometry
\pagestyle{fancy} % options: empty , plain , fancy
\renewcommand{\headrulewidth}{0pt} % customise the layout...
\lhead{}\chead{}\rhead{}
\lfoot{}\cfoot{\thepage}\rfoot{}

%%% SECTION TITLE APPEARANCE
\usepackage{sectsty}
\allsectionsfont{\sffamily\mdseries\upshape} % (See the fntguide.pdf for font help)
% (This matches ConTeXt defaults)

%%% ToC (table of contents) APPEARANCE
\usepackage[nottoc,notlof,notlot]{tocbibind} % Put the bibliography in the ToC
\usepackage[titles,subfigure]{tocloft} % Alter the style of the Table of Contents
\renewcommand{\cftsecfont}{\rmfamily\mdseries\upshape}
\renewcommand{\cftsecpagefont}{\rmfamily\mdseries\upshape} % No bold!

\usepackage[hidelinks]{hyperref}
\usepackage{float}

\usepackage[italian]{babel}

\usepackage{biblatex} %Imports biblatex package
\bibliography{citations} %Import the bibliography file

\setlength{\parindent}{0cm}

\usepackage{listings}
\usepackage{color}
\definecolor{lightgray}{rgb}{.9,.9,.9}
\definecolor{darkgray}{rgb}{.4,.4,.4}
\definecolor{purple}{rgb}{0.65, 0.12, 0.82}

\lstdefinelanguage{SWRL}{
  keywords={},
  otherkeywords={^, ->},
  keywordstyle=\color{blue}\bfseries,
  identifierstyle=\color{black},
  sensitive=false,
  breakatwhitespace=true
}

\lstset{
   backgroundcolor=\color{lightgray},
   extendedchars=true,
   basicstyle=\footnotesize\ttfamily,
   showstringspaces=false,
   showspaces=false,
   numbers=none,
   tabsize=2,
   breaklines=true,
   showtabs=false,
   captionpos=b
}

\usepackage{graphicx}
\graphicspath{ {./img/} }

%%% END Article customizations

%%% The "real" document content comes below...

\title{FootOntologyPlus}
\author{
  Giacomo Leo Bertuccioli\\
  \texttt{0001136879}
  \and
  Alex Mazzotti\\
  \texttt{0001146685}
}
\date{} % Activate to display a given date or no date (if empty),
         % otherwise the current date is printed 

\begin{document}
\maketitle

\newpage

\renewcommand*\contentsname{Indice}

\tableofcontents
\newpage

\chapter{Introduzione}

Footology \cite{foot} è un knowledge representation system progettato per rappresentare e organizzare le informazioni rilevanti del mondo del calcio.
L'ontologia consente di modellare i diversi aspetti e concetti del gioco, come ad esempio partite, squadre di calcio e giocatori.
L'idea dietro al progetto è quella di facilitare l'integrazione dei dati, l'interoperabilità e fornire sistemi di interrogazione avanzati per applicazioni riguardanti il calcio.

\hfill

La nostra estensione FootOntologyPlus mira ad espandere l'originale, modellando meglio alcuni dei concetti inclusi da quest'ultima o aggiungendone di nuovi.
L'entità delle nostre modifiche è discussa nei capitoli successivi.

\hfill

Il nostro progetto è stato realizzato usando l'editor \textbf{Protegé}, un programma open source per visualizzare e modificare ontologie.

\chapter{Ontologia di partenza}

\section{Informazioni di base}

Footology è disponibile su Github \cite{foot}, pubblicata dall'utente \textbf{arditb1997}, ed è un'ontologia relativamente semplice che modella gli aspetti principali del calcio.

\hfill

L'ontologia è disponibile in 3 formati diversi: \textbf{owl}, \textbf{rdf} e \textbf{ttl}.
\textbf{OWL} e \textbf{RDF} sono due linguaggi di knowledge representation, mentre il terzo, \textbf{Turtle}, è semplicemente una sintassi usata per esprimere ontologie scritte in uno dei due linguaggi precedenti in un formato testuale più facile da leggere. 

\section{Dettagli dell'ontologia}

La seguente tabella riassume i contenuti dell'ontologia:

\begin{center}
    \begin{tabular}{||c|c||}
        \hline
        Classi & \texttt{13} \\
        \hline
        Object properties & \texttt{26} \\
        \hline
        Data properties & \texttt{46} \\
        \hline
        Individui & \texttt{0} \\
        \hline
        Nodi (del grafo) & \texttt{59} \\
        \hline
        Foglie (del grafo) & \texttt{62} \\
        \hline
    \end{tabular}
\end{center}

\section{Classi dell'ontologia}

Le classi contenute nell'ontologia sono:

\begin{itemize}
    \item \texttt{Player}, che rappresenta un calciatore;
    \item \texttt{Team}, che rappresenta una squadra di calcio;
    \item \texttt{Match}, una partita di calcio;
    \item \texttt{PerformanceStats}, statistiche riguardanti un calciatore in una partita o all'intera partita;
    \item \texttt{Tournament}, un torneo di calcio, che ha come sottoclassi \texttt{KnockOutTournament}, un torneo piramidale ad eliminazione diretta, e \texttt{League}, un torneo in cui ogni squadra affronta tutte le altre e accumula punti in base al risultato delle partite;
    \item \texttt{Stadium}, uno stadio in cui si giocano le partite;
    \item \texttt{Coach}, un allenatore;
    \item \texttt{Referee}, un arbitro;
    \item \texttt{Position}, la posizione di un giocatore in partita;
    \item \texttt{Trophy}, un trofeo vinto da una squadra;
    \item \texttt{Award}, un premio vinto da un giocatore.
\end{itemize}

\section{Proprietà importanti}

Le proprietà più importanti nell'ontologia sono:

\begin{itemize}
    \item \texttt{playsFor}, che collega un giocatore a una squadra in cui gioca;
    \item \texttt{participatesIn}, che collega un giocatore a una partita a cui partecipa;
    \item \texttt{competesIn}, che collega una squadra a una partita in cui è coinvolta;
    \item \texttt{partOf}, che collega una squadra ad un torneo a cui partecipa;
    \item \texttt{hasStats} e \texttt{hasPerformanceStats}, che collegano rispettivamente un giocatore e una partita a delle statistiche;
    \item \texttt{hosts}, che collega uno stadio ad una partita;
    \item \texttt{officiates}, che collega un arbitro ad una partita;
    \item \texttt{manages}, che collega un allenatore ad una squadra.
\end{itemize}
    
\begin{figure}[H]
    \begin{center}
        \includegraphics[width=0.85\textwidth]{footology_2.1.png}
    \end{center}
    \caption{Grafico dell'ontologia, realizzato con WebVOWL \cite{webvowl,foot}.}
\end{figure}
\begin{figure}[H]
    \begin{center}
        \includegraphics[width=0.85\textwidth]{footology_v2.1_schema.png}
    \end{center}
    \caption{Schema dell'ontologia, realizzato con Draw.io \cite{drawio, foot}.}
\end{figure}
    
\section{Ontologie incluse}

Per realizzare l'ontologia sono stati utilizzati:

\begin{itemize}
    \item \textbf{RDF} (\textbf{Resource Description Language}), un modello per la rappresentazione di dati (e metadati), avente una struttura graph-based nella quale essi sono scritti sotto forma di triple soggetto-predicato-oggetto;
    \item \textbf{RDFS} (\textbf{RDF Schema}), un insieme di classi e proprietà utili che estendono il vocabolario di RDF;
    \item \textbf{OWL} (\textbf{Web Ontology Language}), un linguaggio standard usato per realizzare ontologie, che estende RDF;
    \item \textbf{DublinCore}, un vocabolario che contiene termini standard per la definizione di metadati (titolo, autore, descrizione, ecc.).
\end{itemize}

\chapter{Modifiche all'ontologia}
L'ontologia iniziale aveva una struttura minimale e mancava di molti concetti.
È stato necessario eseguire un lavoro di ristrutturazione, per renderla più completa possibile, e adeguata al contesto del calcio.

\hfill

In particolare, sono state aggiunte nuove informazioni tramite l’introduzione di classi e relazioni, è stata modificata la struttura complessiva dell’ontologia per migliorarne l’organizzazione, e infine si è cercato di integrarla con parti esistenti di altre ontologie, attraverso operazioni di collegamento (join) tra concetti compatibili.
 \newpage
 \section{Modifica concetti esistenti}

\begin{itemize}
    \item \textbf{Match}

    In \texttt{Match} è stato aggiunta come sottoclasse la  classe \texttt{FriendlyMatch} che rappresenta le partite amichevoli, ovvere quelle che non appartengono a un \texttt{Tournament}.
    \begin{figure}[H]
        \centering
        \includegraphics[width=0.3\linewidth]{MATCH.png}
        \caption{UML Match}
    \end{figure}
    \item \textbf{PerformanceStats}

    Invece, in \texttt{PerformanceStats} sono state aggiunte due sottoclassi, disgiunte tra di loro, \texttt{PlayerPerformanceStats}, e \texttt{TeamPerformanceStats}. Questo è stato fatto per distinguere meglio il concetto di statistiche su un \texttt{Match}, e statistiche di un singolo \texttt{Player}.
    Per esempio la statistica del possesso palla non è concetto che è associato a un singolo calciatore ma è associata alla squadra.
    \begin{figure}[H]
        \centering
        \includegraphics[width=0.4\linewidth]{STATS.png}
        \caption{UML PerformanceStats}
    \end{figure}
    \item \textbf{Position}

    In \texttt{Position}, sono state definite le  specifiche posizioni in campo, così da rappresentare in modo preciso e strutturato il ruolo dei giocatori all'interno della formazione. Nell'ontologia originale erano modellate come Data Properties.
    \begin{figure}[H]
        \centering
        \includegraphics[width=1\linewidth]{POSITION.png}
        \caption{UML Position}
    \end{figure}
    \hspace{2cm}

    Le abbreviazioni CB, FB, WB, CDM, etc indicano le specifiche posizioni in campo, per esempio CB significa Central Back, cioè Difensore centrale.
 \end{itemize}
\newpage

\section{Join con altre ontologie}

Come richiesto dalla consegna, è stato necessario integrare l'ontologia con altre ontologie esistenti, facendo operazioni di join per favorire l'interoperabilità dei concetti.

In particolare, sono stati importati:

\begin{itemize}
    \item La classe \texttt{Person} dall'ontologia \textit{schema.org} \cite{schema}, utilizzata per rappresentare entità individuali come giocatori, allenatori e arbitri. 
    
    Nella nostra ontologia, \texttt{Person} è stata inserita come superclasse delle classi \texttt{Player}, \texttt{Coach} e \texttt{Referee}. Il join è stato eseguito in modo da includere solo le Object Properties che ci interessavano, escludendo quelle non rilevanti per il nostro dominio.

    \begin{figure}[H]
        \centering
        \includegraphics[width=0.8\linewidth]{PERSON.png}
        \caption{UML Person}
    \end{figure}
    \newpage
    \item Le classi \texttt{City} e \texttt{Country} da \textit{dbpedia.org} \cite{dbpedia}, adottate per modellare i concetti geografici. 
  
    Anche in questo caso è stato effettuato un join selettivo: \texttt{City} e \texttt{Country} sono state collegate alla nostra ontologia per rappresentare, rispettivamente, la città dello stadio, la nazione della lega, e la nazionalità delle persone.
    
    La relazione \texttt{hasNationality} tra \texttt{Country} e \texttt{Person}, è stata definita direttamente sulla superclasse \texttt{Person}, in modo da poter utilizzare tale relazione per tutte le sue sottoclassi, ovvero \texttt{Player}, \texttt{Referee} e \texttt{Coach}.
\end{itemize}

\newpage


 \section{Classi aggiunte}

Come detto a inizio capitolo l'ontologia di partenza era incompleta su vari aspetti. Di seguito, si mostrano i nuovi concetti inseriti.

\begin{itemize}[leftmargin=*]
    \item \texttt{Contract}, con sottoclassi \texttt{CoachContract} e \texttt{PlayerContract}

    I contratti sono utili per mettere in relazione giocatori e allenatori alle squadre, tenendo conto anche di una storicizzazione con i contratti terminati. Per esempio un contratto terminato, indica che un certo giocatore/allenatore ha giocato per una certa squadra.
  
    \begin{figure}[H]
        \centering
        \includegraphics[width=0.8\linewidth]{CONTRACT.png}
        \caption{UML contratti in relazione con Team, Coach e Player}
    \end{figure}

    Nella figura 3.5 sono mostrate solo alcune relazioni che legano i contratti con giocatori e allenatori. Per esempio non sono mostrate le inverse, o la relazione padre,  \texttt{Person} - \texttt{Contract}, e \texttt{Team} - \texttt{Contract}. 

    \newpage
    
    \item \texttt{Formation}
    
    Inizialmente, i moduli adottati dalle squadre nei match erano rappresentati tramite Data Properties. Tuttavia, poiché l’insieme dei moduli più comuni è ben definito (es. 4-3-3, 3-5-2, ecc.), abbiamo preferito modellare i moduli come sottoclassi di \texttt{Formation}.

    \item \texttt{MatchEvent}
    
    Gli eventi dei match non erano presenti, quindi li abbiamo inseriti attraverso una gerarchia di classi.

    \begin{figure} [h]
        \centering
        \includegraphics[width=\textwidth]{MATCHEVENT.png}
        \caption{UML eventi Match}
    \end{figure}

    Nella figura 3.6 sono mostrate anche come gli eventi sono messi in relazione ai \texttt{Match}; è stata inserita anche la relazione ai \texttt{Team}, per modellare di quale squadra è l'evento. Inoltre alcuni eventi sono associati ai \texttt{Player}. 

\end{itemize}

\newpage
\subsection{Classi d'errore}
Per gestire eventuali errori dovuti a incompatibilità nei dati, rilevati tramite regole SWRL, è stata introdotta una classe \texttt{Error}. Questa classe raccoglie diverse sottoclassi, ciascuna delle quali rappresenta un tipo specifico di errore semantico o logico. In questo modo, è possibile classificare e tracciare gli errori direttamente all’interno dell’ontologia, facilitando il controllo di consistenza durante la validazione dei dati.

Le sottoclassi di \texttt{Error} sono: 
\begin{itemize}
    \item \texttt{FriendlyMatchInTournamentError}
    \item \texttt{InconsistentBallPossessionInMatchError}
    \item \texttt{InconsistentContractDatesError}
    \item \texttt{InconsistentMatchScoresError}
    \item \texttt{InconsistentPlayerPassesError}
    \item \texttt{InvalidLeagueMatchFormatError}
    \item \texttt{InvalidMatchError}
    \item \texttt{InvalidSubstitutionError}
    \begin{itemize}
        \item \texttt{DuplicateSubstitutedError}
        \item \texttt{DuplicateSubstituteError}
        \item \texttt{InvalidReserveReplacementError}
        \item \texttt{InvalidSubstitutionPlayersError}
        \item \texttt{StarterAsSubstituteError}
    \end{itemize}
    \item \texttt{MultipleRedCardsError}
    \item \texttt{NonGoalKeeperWithSavesError}
    \item \texttt{TeamNotInMatchError}
    \item \texttt{UncontractedPlayerInMatchError}
\end{itemize}

Vedi capitolo su regole SWRL per l'utilizzo di queste classi.
\newpage

\section{Proprietà aggiunte}
Di seguito una lista delle proprietà più importanti aggiunte.

Object Properties:
\begin{itemize}
    \item \texttt{hasEvent} (da \texttt{Match} a \texttt{MatchEvent}): associa a una partita i suoi eventi (goal, falli, sostituzioni, ecc.).
    \item \texttt{hasFormation} (da \texttt{Team} a \texttt{Formation}): specifica la formazione adottata da una squadra.
    \item \texttt{hasMatchEvent} (da \texttt{Team} a \texttt{MatchEvent}): collega una squadra agli eventi della partita.
    \item \texttt{hasPlayer} (da \texttt{Match} a \texttt{Player}): elenca i giocatori che partecipano a una determinata partita.
    \item \texttt{hasPlayerPerformanceStats} (da \texttt{Match} a \texttt{PlayerPerformanceStats}): collega una partita alle statistiche individuali dei giocatori.
    \item \texttt{hasPosInFormation} (da \texttt{Formation} a \texttt{Position}): definisce le posizioni previste all'interno di una formazione.
    \item \texttt{hasTeamFormation} (da \texttt{Match} a \texttt{Formation}): specifica le formazioni usate dalle squadre in una partita.
    \item \texttt{hasWinner} (da \textit{Match} a \textit{Team}): indica la squadra vincitrice della partita.
    \item \texttt{includes} (da \texttt{Match} a \texttt{Team}): rappresenta le squadre che partecipano a una partita; ha come sotto-proprietà:
    
    \begin{itemize}
        \item \texttt{hasHomeTeam}: indica la squadra di casa;
        \item \texttt{hasAwayTeam}: indica la squadra ospite.
    \end{itemize}

    \item \texttt{involves} (da \texttt{Contract} a \texttt{Player} o \texttt{Coach}): collega un contratto con un giocatore o un allenatore.
    \item \texttt{partecipatesIn} (da \texttt{Player} a \texttt{Match}): indica le partite a cui ha partecipato un giocatore.
    \item \texttt{scores} (da \texttt{Player} a \texttt{Goal}): rappresenta i goal segnati da un giocatore.
    \item \texttt{signs} (da \texttt{Team} a \texttt{Contract}): indica i contratti firmati da una squadra.
    \item \texttt{statsIn} (da \texttt{PerformanceStats} a \texttt{Match}): collega le statistiche a una specifica partita.
\end{itemize}

\newpage

Data Properties:

\begin{itemize}
    \item \texttt{BallPossession} (su \texttt{TeamPerformanceStats}): indica la percentuale di possesso palla di una squadra in una partita.
    \item \texttt{BirthDate} (su \texttt{Person}): rappresenta la data di nascita di una persona.
    \item \texttt{IsFriendlyMatch} (su \texttt{Match}): specifica se una partita è amichevole.
    \item \texttt{MatchAwayTeamScore} (su \texttt{Match}): indica il numero di goal segnati dalla squadra ospite.
    \item \texttt{MatchHomeTeamScore} (su \texttt{Match}): indica il numero di goal segnati dalla squadra di casa.
    \item \texttt{MatchDate} (su \texttt{Match}): rappresenta la data in cui si è svolta la partita.
    \item \texttt{MinuteOfEvent} (su \texttt{MatchEvent}): indica il minuto della partita in cui si è verificato un evento.
    \item \texttt{NumberOfDefenders} (su \texttt{Formation}): specifica il numero di difensori nella formazione.
    \item \texttt{NumberOfMidfielders} (su \texttt{Formation}): specifica il numero di centrocampisti nella formazione.
    \item \texttt{NumberOfStrikers} (su \texttt{Formation}): specifica il numero di attaccanti nella formazione.
\end{itemize}

\begin{figure}[H]
    \begin{center}
        \includegraphics[width=0.75\textwidth]{FootOntologyPlus.rdf.png}
    \end{center}
    \caption{Grafico completo dell'ontologia modificata, realizzato con WebVOWL \cite{webvowl,foot}.}
\end{figure}

\chapter{Inserimento di individui}
La fase successiva ha riguardato l'inserimento degli individui nelle varie classi. L'obiettivo era quello di aggiungere esempi concreti di istanze all'interno dell'ontologia, così da renderla effettivamente utilizzabile per interrogazioni \textbf{SPARQL}.

\hfill

Creare tutti gli individui per ogni classe e definire tutte le relative relazioni è un'attività molto lunga e ripetitiva. Per questo motivo, ci siamo limitati a creare una sottoparte degli individui, sufficiente per mostrare il funzionamento delle query.

\hfill

Per velocizzare questa fase, abbiamo utilizzato \textbf{Cellfie}, un plugin di \textbf{Protegé} che consente di generare individui a partire da un file Excel.
Nel nostro caso, ogni riga del file Excel rappresenta un'istanza (individuo) da creare, mentre le colonne contengono le informazioni associate a questo. In particolare:

\begin{itemize}
    \item la prima colonna contiene il nome dell'individuo;
    \item le altre colonne rappresentano i valori associati a una o più Object Properties e/o Data Properties
\end{itemize}
\newpage
\section{Esempio inserimento con Cellfie}
Di seguito è riportato come abbiamo inserito molti contratti, presenti in un file Excel, attraverso il plugin \textbf{Cellfie} e come abbiamo impostato le regole di trasformazione necessarie a trasformare i dati grezzi in individui e relazioni vere e proprie.

\hfill

Nella Figura 4.1 è mostrata la tabella da inserire, mentre nell'immagine 4.2 si può osservare come sono utilizzati i campi di ogni colonna.
In questo caso abbiamo:
\begin{itemize}
    \item Colonna A: rappresenta il nome dell'individuo
    \item Colonna B: Data Properties per l'inizio del contratto
    \item Colonna C: Data Properties per la fine del contratto (se il contratto è ancora attivo allora non è presenta la data di fine contratto.)
    \item Colonna D: Data Properties per il valore del contratto
    \item Colonna E: Object Properties della squadra a cui è associato
    \item Colonna F: Object Properties del giocatore a cui è associato
\end{itemize}

\hfill

\begin{figure}[H]
    \centering
    \includegraphics[width=0.9\linewidth]{EXCELCONTRACT.jpeg}
    \caption{Foglio Excel dei contratti}
\end{figure}

\begin{figure}[H]
    \centering
    \includegraphics[width=0.9\linewidth]{RULECELLFIE.jpeg}
    \caption{Regola di trasformazione per creare i contratti presenti nel foglio Excel}
\end{figure}

\chapter{Regole SWRL}

\textbf{SWRL} (\textbf{Semantic Web Rule Language}) è un linguaggio usato nell'ambito del Semantic Web per esprimere regole di inferenza e vincoli logici più complessi di quelli realizzabili usando il solo OWL.

\hfill

Nella nostra ontologia abbiamo inserito numerose regole, utili a:

\begin{enumerate}
    \item inferire nuovi collegamenti tra i dati;
    \item individuare casi di errore sui dati.
\end{enumerate}

Le sezioni che seguono riportano tali regole, non in un ordine particolare.

\section{HomeFormationRule}

Questa regola "raffina" il collegamento tra una formazione e una partita, esplicitando che la prima è usata dalla squadra in casa.
Questo perché \texttt{isAwayFormationIn} è una sottoproprietà di \texttt{isFormationIn}.

\begin{lstlisting}[language=SWRL]
isHomeTeam(?t, ?m) ^ hasFormation(?t, ?f) ^ isFormationIn(?f, ?m) -> isHomeFormationIn(?f, ?m)
\end{lstlisting}

\section{AwayFormationRule}

Questa regola funziona in modo analogo alla precedente, ma riguardo la squadra in trasferta.

\begin{lstlisting}[language=SWRL]
isAwayTeam(?t, ?m) ^ hasFormation(?t, ?f) ^ isFormationIn(?f, ?m) -> isAwayFormationIn(?f, ?m)
\end{lstlisting}

\newpage

\section{DuplicateSubstituteRule}

Questa regola verifica se un giocatore segnato come riserva partecipa a multiple sostituzioni come giocatore entrante, e in tale caso gli assegna una classe di errore.

\begin{lstlisting}[language=SWRL]
hasReservePlayer(?f, ?p) ^ hasEnteringPlayer(?s1, ?p) ^ hasEnteringPlayer(?s2, ?p) ^ differentFrom(?s1, ?s2) ^ hasTeamFormation(?m, ?f) ^ hasSubstitution(?m, ?s1) ^ hasSubstitution(?m, ?s2) -> DuplicateSubstituteError(?p)
\end{lstlisting}

\section{DuplicateSubstitutedRule}

Il funzionamento di questa regola è analogo alla precedente, ma riguardante il giocatore uscente.
Questa regola copre però solo i giocatori titolari (ovvero i giocatori che partecipano alla partita dall'inizio).

\begin{lstlisting}[language=SWRL]
hasStarterPlayer(?f, ?p) ^ hasExitingPlayer(?s1, ?p) ^ hasExitingPlayer(?s2, ?p) ^ differentFrom(?s1, ?s2) ^ hasTeamFormation(?m, ?f) ^ hasSubstitution(?m, ?s1) ^ hasSubstitution(?m, ?s2) -> DuplicateSubstitutedError(?p)
\end{lstlisting}

\section{StarterAsSubstituteRule}

Questa regola verifica se un giocatore titolare è il giocatore entrante in una sostituzione, e in tale caso gli assegna una classe di errore.

\begin{lstlisting}[language=SWRL]
hasStarterPlayer(?f, ?p) ^ hasEnteringPlayer(?s, ?p) ^ hasTeamFormation(?m, ?t) ^ hasSubstitution(?m, ?s) -> StarterAsSubstituteError(?p)
\end{lstlisting}

\section{FriendlyMatchInTournamentRule}

Non è corretto assegnare una partita amichevole ad un torneo, qualsiasi sia il suo tipo, e se ciò viene fatto allora una classe di errore è assegnata alla partita. 

\begin{lstlisting}[language=SWRL]
includedInTournament(?m, ?t) ^ FriendlyMatch(?m) -> FriendlyMatchInTournamentError(?m)
\end{lstlisting}

\section{GoalScoredByTeamRule}

Questa regola collega un goal ad una squadra, sapendo che il goal è stato fatto da un giocatore che giocava in tale squadra nella partita.

\begin{lstlisting}[language=SWRL]
scores(?p, ?g) ^ hasEvent(?m, ?g) ^ isPlayerInFormation(?p, ?f) ^ hasFormation(?t, ?f) ^ competesIn(?t, ?m) -> scoredByTeam(?g, ?t)
\end{lstlisting}

\section{InconsistentBallPossessionInMatchRule}

Questa regola controlla se la somma del possesso palla indicata nelle statistiche di performance per le squadre che hanno giocato una partita non è 100\%, e in tale caso assegna una classe di errore alla partita e alle statistiche.

\begin{lstlisting}[language=SWRL]
teamStatsIn(?ps1, ?m) ^ teamStatsIn(?ps2, ?m) ^ differentFrom(?ps1, ?ps2) ^ BallPossession(?ps1, ?p1) ^ BallPossession(?ps2, ?p2) ^ swrlb:add(?r, ?p1, ?p2) ^ swrlb:notEqual(?r, 100) -> InconsistentBallPossessionInMatchError(?m) ^ InconsistentBallPossessionInMatchError(?ps1) ^ InconsistentBallPossessionInMatchError(?ps2)
\end{lstlisting}

\section{InconsistentContractDatesRule}

Questa regola verifica se le date di un contratto sono inconsistenti (data di inizio futura alla data di fine).

\begin{lstlisting}[language=SWRL]
ContractStartDate(?c, ?sd) ^ ContractEndDate(?c, ?ed) ^ temporal:before(?ed, ?sd) -> InconsistentContractDatesError(?c)
\end{lstlisting}

\section{InconsistentMatchScoresHome/AwayRule}

Queste due regole verificano se il vincitore di una partita non è consistente con i punteggi assegnati alle due squadre.
Sono state necessarie due regole in quanto i punteggi delle squadre sono legati alla partita e non direttamente alle squadre.

\begin{lstlisting}[language=SWRL]
isHomeTeam(?t1, ?m) ^ isAwayTeam(?t2, ?m) ^ hasWinner(?m, ?t1) ^ MatchHomeTeamScore(?m, ?s1) ^ MatchAwayTeamScore(?m, ?s2) ^ swrlb:lessThanOrEqual(?s1, ?s2) -> InconsistentMatchScoresError(?m)
\end{lstlisting}

\begin{lstlisting}[language=SWRL]
isHomeTeam(?t1, ?m) ^ isAwayTeam(?t2, ?m) ^ hasWinner(?m, ?t2) ^ MatchHomeTeamScore(?m, ?s1) ^ MatchAwayTeamScore(?m, ?s2) ^ swrlb:lessThanOrEqual(?s2, ?s1) -> InconsistentMatchScoresError(?m)
\end{lstlisting}

\section{InvalidLeagueMatchFormatExtraTime/PenaltyRule}

Le partite che fanno parte di una lega non possono avere supplementari o rigori, e queste due regole servono a identificare tali inconsistenze.

\begin{lstlisting}[language=SWRL]
includedInTournament(?m, ?t) ^ League(?t) ^ ExtraTimePlayed(?m, true) -> InvalidLeagueMatchFormatError(?m)
\end{lstlisting}

\begin{lstlisting}[language=SWRL]
includedInTournament(?m, ?t) ^ League(?t) ^ PenaltyShootoutPlayed(?m, true) -> InvalidLeagueMatchFormatError(?m)
\end{lstlisting}

\section{InvalidMatchRule}

Questa regola verifica se una squadra sta giocando contro sé stessa in una partita.

\begin{lstlisting}[language=SWRL]
hasHomeTeam(?m, ?t) ^ hasAwayTeam(?m, ?t) -> InvalidMatchError(?m)
\end{lstlisting}

\section{MultipleRedCardsRule}

Questa regola controlla se un giocatore ha ricevuto multipli cartellini rossi in una partita.

\begin{lstlisting}[language=SWRL]
hasReceived(?p, ?c1) ^ hasReceived(?p, ?c2) ^ RedCard(?c1) ^ RedCard(?c2) ^ differentFrom(?c1, ?c2) ^ hasEvent(?m, ?c1) ^ hasEvent(?m, ?c2) -> MultipleRedCardsError(?p)
\end{lstlisting}

\section{NonGoalKeeperWithSavesRule}

Questa regola verifica se le statistiche di un giocatore indicano il numero di parate sebbene esso non abbia giocato come portiere nella partita.

\begin{lstlisting}[language=SWRL]
playsInPosition(?pl, ?p) ^ Outfield(?p) ^ hasPlayerStats(?pl, ?ps) ^ Saves(?ps, ?s) ^ hasPlayer(?m, ?p) ^ hasPlayerPerformanceStats(?m, ?ps) -> NonGoalKeeperWithSavesError(?pl)
\end{lstlisting}

\section{PlayerInMatchRule}

Questa regola collega un giocatore ad una partita, sapendo che il giocatore era in una formazione utilizzata in tale partita.

\begin{lstlisting}[language=SWRL]
hasPlayerInFormation(?f, ?p) ^ isFormationIn(?f, ?m) -> hasPlayer(?m, ?p)
\end{lstlisting}

\section{TeamNotInMatchRule}

Questa regola controlla se un evento in una partita è legato ad una squadra che non gioca in tale partita.

\begin{lstlisting}[language=SWRL]
hasEvent(?m, ?e) ^ hasMatchEvent(?t, ?e) ^ isHomeTeam(?t1, ?m) ^ isAwayTeam(?t2, ?m) ^ differentFrom(?t, ?t1) ^ differentFrom(?t, ?t2) -> TeamNotInMatchError(?t)
\end{lstlisting}

\newpage

\section{UncontractedPlayerInMatchRule}

Questa regola verifica se un giocatore ha partecipato ad una partita senza far parte di una delle due squadre coinvolte, in quanto non si trovava sotto contratto con una delle due squadre al momento della partita.
La regola copre soltanto i casi in cui il giocatore si trovava sotto contratto con un'altra squadra e se il contratto è terminato.

\begin{lstlisting}[language=SWRL]
signs(?t1, ?c) ^ involvesPlayer(?c, ?p) ^ participatesIn(?p, ?m) ^ includes(?m, ?t2) ^ includes(?m, ?t3) ^ differentFrom(?t1, ?t2) ^ differentFrom(?t1, ?t3) ^ ContractStartDate(?c, ?sd) ^ ContractEndDate(?c, ?ed) ^ MatchDate(?m, ?d) ^ temporal:before(?sd, ?d) ^ temporal:after(?ed, ?d) -> UncontractedPlayerInMatchError(?p)
\end{lstlisting}
\chapter{Interrogazioni SPARQL}

L'ultima parte del progetto è stata la realizzazione di interrogazioni usando \textbf{SPARQL} (\textbf{SPARQL Protocol and RDF Query Language}), un linguaggio simile a SQL per poter estrarre dati dall'ontologia.

\hfill

Le interrogazioni che abbiamo scritto sono esempi che producono risultati sulla base degli individui d'esempio che abbiamo predisposto. 
Per realizzarle abbiamo usato le viste \textbf{SPARQL Query} e \textbf{Snap SPARQL Query} di \textbf{Protegé}. 

\section{Interrogazione sui contratti di un giocatore}

Questa interrogazione recupera i contratti firmati da un giocatore (nell'esempio \texttt{Player1}) e i nomi dei team per cui i contratti sono stati firmati. 

\begin{lstlisting}
PREFIX : <http://visualdataweb.org/FootOntologyPlus/>

SELECT ?contract ?teamName
WHERE { 
    ?team :signsPlayer ?contract ;
          :TeamName ?teamName .
    ?contract :involvesPlayer :Player1 .
}
\end{lstlisting}

\begin{figure}[H]
	\includegraphics[width=\textwidth]{query1}
	\caption{Risultato della query sulla nostra ontologia.}
\end{figure}

\newpage

\section{Interrogazione sui contratti non terminati}

Questa interrogazione recupera tutti i contratti che non hanno una data di terminazione. Si assume in questo caso che tali contratti siano ancora attivi, ma in generale vige la "Open World assumption" riguardo alle informazioni mancanti.

\begin{lstlisting}
PREFIX : <http://visualdataweb.org/FootOntologyPlus/>

SELECT ?player ?team ?contract ?startDate
WHERE { 
    ?contract :isSignedForPlayerBy ?team ;
        :involvesPlayer ?player ;
        :ContractStartDate ?startDate .
    FILTER NOT EXISTS { ?contract :ContractEndDate ?endDate }
}
ORDER BY ASC (?player)
\end{lstlisting}

\begin{figure}[H]
	\includegraphics[width=\textwidth]{query2}
	\caption{Parte del risultato della query sulla nostra ontologia.}
\end{figure}

\section{Interrogazione sui cartellini ricevuti dai giocatori}

Questa interrogazione recupera le informazioni riguardanti i cartellini ricevuti dai giocatori, nello specifico il tipo di cartellino, la partita e il minuto di gioco in cui è stato ricevuto. 

\begin{lstlisting}
PREFIX rdf: <http://www.w3.org/1999/02/22-rdf-syntax-ns#>
PREFIX rdfs: <http://www.w3.org/2000/01/rdf-schema#>
PREFIX : <http://visualdataweb.org/FootOntologyPlus/>

SELECT ?player ?cardType ?match ?minute
WHERE {
    ?player :hasReceived ?card .
    ?card :MinuteOfEvent ?minute ;
        :isEventIn ?match ;
        rdf:type ?cardType .
    ?cardType rdfs:subClassOf :Card .
    FILTER (?cardType != :Card) .
}
ORDER BY ASC (?player)
\end{lstlisting}

\begin{figure}[H]
	\includegraphics[width=\textwidth]{query3}
	\caption{Risultato della query sulla nostra ontologia.}
\end{figure}

\section{Interrogazione sulle sostituzioni effettuate in una partita}

Questa interrogazione le informazioni riguardanti le sostituzioni effettuate in una partita (nell'esempio \texttt{Match1}), nello specifico il nome dei giocatori coinvolti e il minuto di gioco. 

\begin{lstlisting}
PREFIX : <http://visualdataweb.org/FootOntologyPlus/>

SELECT ?enteringName ?exitingName ?minute
WHERE {
    ?sub :isSubstitutionIn :Match1 ;
        :hasEnteringPlayer ?entering ;
        :hasExitingPlayer ?exiting ;
        :MinuteOfEvent ?minute .
    ?entering :FullName ?enteringName .
    ?exiting :FullName ?exitingName .
}
ORDER BY ASC (?minute)
\end{lstlisting}

\begin{figure}[H]
	\includegraphics[width=\textwidth]{query4}
	\caption{Risultato della query sulla nostra ontologia.}
\end{figure}

\section{Interrogazione sui goal segnati dai giocatori}

Questa interrogazione recupera il numero di goal segnati da ogni giocatore in un torneo (nell'esemptio \texttt{League1}).

\begin{lstlisting}
PREFIX : <http://visualdataweb.org/FootOntologyPlus/>

SELECT ?player ?playerName (COUNT(?goal) as ?totalGoals) 
WHERE {
    ?match :includedInTournament :League1 ;
        :hasGoal ?goal .
    ?goal :scoredByPlayer ?player .
    ?player :FullName ?playerName
}
GROUP BY ?player ?playerName
ORDER BY DESC (?totalGoals)
\end{lstlisting}

\begin{figure}[H]
	\includegraphics[width=\textwidth]{query5}
	\caption{Risultato della query sulla nostra ontologia. Il nome dell'individuo giocatore è incluso in caso di omonimi.}
\end{figure}

\chapter{Confronto metriche}

Di seguito un confronto delle metriche dell'ontologia di partenza "Footology" e quella finale "FootOntologyPlus".

\hfill

\begin{table}[h!]
\centering
\begin{tabular}{|l|c|c|}
\hline
\textbf{Metrica} & \textbf{Footology} & \textbf{FootOntologyPlus} \\
\hline
Axioms & 496 & 2.160 \\
Logical axiom count & 180 & 1.530 \\
Declaration axioms count & 88 & 437 \\
Class count & 13 & 81 \\
Object property count & 26 & 123 \\
Data property count & 46 & 51 \\
Individual count & 0 & 180 \\
Annotation Property count & 5 & 7 \\
\hline
\end{tabular}
\caption{Confronto metriche}
\end{table}

\hfill

Osservando la tabella si può vedere il lavoro di ristrutturazione e sopratutto di estensione dell'ontologia di partenza.


\chapter{Conclusioni}

In conclusione, il nostro progetto ci ha permesso di lavorare concretamente su una vera e propria ontologia, mettendo in pratica i concetti appresi durante il corso. L’obiettivo principale è stato quello di rendere l’ontologia il più possibile completa, coerente e adatta a rappresentare in modo accurato il dominio del calcio.

\hfill

L'ontologia di partenza era molto limitata e incompleta, il che ci ha dato l'opportunità di intervenire in modo significativo con numerose modifiche e aggiunte. Come mostrano le metriche del capitolo precedenti, il numero di classi è passato da 13 a 81, le Object Properties da 26 a 123, etc... questo ci dice quanto è stata ampliata l'ontologia.

\hfill

Questo progetto ci ha inoltre permesso di comprendere l'importanza delle ontologie come strumenti fondamentali per strutturare e interrogare grandi quantità di dati in modo efficiente, verificabile e interoperabile.

\hfill

Sebbene l’espansione realizzata sia focalizzata su uno scenario accademico, il lavoro svolto costituisce una base solida per possibili estensioni future e per l’utilizzo dell’ontologia in applicazioni reali, come sistemi di analisi calcistica, motori di raccomandazione, o strumenti per la gestione e validazione di dataset sportivi complessi.


\printbibliography

\end{document}
